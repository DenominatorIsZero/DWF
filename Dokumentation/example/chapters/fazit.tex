% !TEX root = ../termpaper.tex
% first example sections
% @author Thomas Lehmann
%

\section{Fazit}
Im Rahmen dieses Projektes wurde eine Oktav-Filterbank mithilfe von Phython implementiert und anschließend simuliert. Hierbei wurde die Oktav-Filterbank mit neun Teilbändern realisiert. Die einzelnen Teilbandfilter bestehen dabei aus bireziproken Wellendigitalfiltern. Hierzu wurde auf das Filterdesign aus \cite{gaszi1983} und \cite{kunold1989} verwendet.\par
Zunächst wurde dafür der bireziproke Wellendigitalfilter implementiert und anschließend zu der Oktav-Filterbank zusammengesetzt. Hierbei mussten die einzelnen Signallaufzeiten ausgeglichen werden, um am Ausgang ein näherungsweise unverfälschtes Signal zu erhalten.\par
Abschließend wurde die Oktav-Filterbank simuliert und analysiert. Hierbei ist zu erkennen, dass die Filterbank am Ausgang an den Übergängen der einzelnen Teilbänder das Signal leicht dämpft.
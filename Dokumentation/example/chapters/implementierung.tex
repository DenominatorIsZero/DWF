\section{Implementierung}
In diesem Kapitel wird die Implementierung der hochselektiven Filterbank in Python 3.7 beschrieben.

\subsection{Übersicht}\label{sec:impl_ueber}

\subsubsection{Verwendete Bibliotheken}\label{sec:impl_bib}
Für einige Berechnungen und das Speichern von Datenstruckturen wurde das Numpy Modul der Scipy Bibliothek~\cite{scipy} verwendet. Für das Erstellen der Diagramme wurde die Matplotlib Bibliothek~\cite{Hunter:2007}.
\subsubsection{Klassendiagramm}\label{sec:impl_klass}
\begin{figure}
  \centering
  \includegraphics[width=1\textwidth]{img/klassendia}
  \caption{Klassendiagramm für die Implementierung einer hochselektiven Filterbank}\label{fig:impl_klassdia}
\end{figure}
Wie in Abbildung~\ref{fig:impl_klassdia} zu sehen erfolgt die Implementierung der hochselektiven Filterbank objektorientiert. Dabei besteht eine Filterbank aus 8 Filtern (vgl. Abb. TODO) und ein Filter aus 9 Elementen (vgl. Abb.).
\subsection{Klassen}\label{sec:impl_klassen}
Im folgenden wird die Implementierung der Klassen im einzelnen beschrieben bevor im nächsten Unterkaptitel auf dei verwendeten Testfunktionen eingegangen wird.
\subsubsection{Element}\label{sec:impl_ele}
Ein Element (s. Lst.~\ref{lst:element}) stellt den kleinsten Teil der Filterbank dar. Neben Ein- und Ausgang verfügt ein Element über ein Delay-Array dessen Länge bei der Initialisierung übergeben werden kann. Hierüber lässt sich die Ordnung des Teilfilters einstellen (vlg. Kap. TODO). Außerdem muss bei der Initialisierung ein Wert für \emph{a} übergeben werden. Dieser bestimmt mit welchem Wert im Element multipliziert wird (vlg. Abb. TODO).

Über das Aufrufen der Funktion \emph{update()} werden die Summen und der Ausgang des Elements neu berechnet. Die Funktion \emph{advance()} wird verwendet um einen Takt des Systems zu simulieren. Der Pointer, welcher den aktuellen Eingang des Verzögerungsgliedes anzeigt wird inkrementiert und ein neuer Wert wird in das Array geschrieben. Der aktuell gültige Augang des Verzögerungsarrays befindet sich eine Stelle \emph{vor} dem Eingang (vgl. Lst.~\ref{lst:element} Zeile 16).

\lstinputlisting[language=Python, firstline=6, lastline=31, caption={Quellcode der Klasse Element}, label={lst:element}]{list/wellendigitalfilter.py}


\subsubsection{Filter}\label{sec:impl_Filter}
Ein Teilfilter (s. Lst.~\ref{lst:filter}) besteht aus 9 Elementen und einem Verzögerungsglied (vlg. Abb. TODO). Bei der Initialisierung muss kann die Anzahl der Verzögerungen und damit die Ordnung des Teilfilters übergeben werden. Die Koeffizienten der Elemnte gemäß~\cite{gaszi1983} über die Funktion \emph{calculate\_gamma(fs, F)} berechnet. Wobei \emph{F} die Taktfrequenz des Systems und \emph{fs} die gewünschte Stopfrequenz des Filters ist.

Ü
\lstinputlisting[language=Python, firstline=34, lastline=118, caption={Quellcode der Klasse Filter}, label={lst:filter}]{list/wellendigitalfilter.py}

\subsubsection{Filterbank}\label{sec:impl_bank}
\lstinputlisting[language=Python, firstline=121, lastline=160, caption={Quellcode der Klasse Filterbank}, label={lst:bank}]{list/wellendigitalfilter.py}
\subsection{Testfunktionen}\label{sec:impl_test}

\subsubsection{Test Filter}\label{sec:impl_testFilter}
\lstinputlisting[language=Python, firstline=163, lastline=193, caption={Quellcode der Testfunktion für einen einzlnen Filter}, label={lst:test_filter}]{list/wellendigitalfilter.py}
\subsubsection{Test Filterbank}\label{sec:impl_testBank}
\lstinputlisting[language=Python, firstline=196, lastline=236, caption={Quellcode der Testfunktion für die gesamte Filterbank}, label={lst:test_bank}]{list/wellendigitalfilter.py}
%%% Local Variables:
%%% mode: latex
%%% TeX-master: "../termpaper"
%%% End:
